\paragraph{}
In this article, we presented both the \emph{principal component analysis} and the \emph{linear
discriminant analysis} methods, which are very useful as pre-processing steps for data analysis. We
have shown that their standard definitions are limited in several ways. We have improved the
methods using the kernel trick.

It must be noted that it makes sense to use both the PCA and the LDA on a dataset, since they are
complementary methods. The main difference between both is that the PCA doesn't require the labeling
of the dataset to be known, while the LDA does.
