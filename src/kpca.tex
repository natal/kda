\paragraph{}
Regarding the PCA, we must first center the data vectors in kernel-space by subtracting the global mean in kernel-space $\lambda$ to each one.
Therefore, the first step is to compute $\lambda$:

$$\lambda = \frac{1}{N}\sum_{i = 1}^{N} \varphi(x_i)$$

This formula is intractable for the reasons mentioned previously. Fortunately, since we want to
compute $\varphi(x_i) - \lambda$ for every $x_i$, it is sufficient to be able to compute a
\emph{distance} in kernel-space. We achieve this as follows:

\begin{align*}
\mathcal{D}(\varphi(x_1), \varphi(x_2)) &= \|\varphi(x_1) - \varphi(x_2)\|^2 \\
&= k(x_1, x_1) - 2k(x_1, x_2) +
k(x_2, x_2)
\end{align*}


\paragraph{}
By replacing $\varphi(x_2)$ by $\lambda$ in the previous equation, we obtain:

$$\mathcal{D}(\varphi(x), \lambda) = k(x, x) - \underbrace{\frac{2}{N}\sum_{i = 1}^N k(y,
x_i)}_{\text{mean of a line of G}} + \underbrace{\frac{1}{N^2}\sum_{i = 1}^{N}\sum_{j = 1}^{N} k(x_i, x_j)}_{\text{mean
of G}}$$

\paragraph{}
Let $\Phi$ be the matrix that contains all the dataset vectors centered in
kernel-space. The $i$th column of matrix $\Phi$ is:

$$(\Phi)_i = [\varphi(x_i) - \lambda] = [\mathcal{D}(\varphi(x_i), \lambda)]$$

\paragraph{} The covariance matrix in kernel-space can now be expressed as:

$$S = \Phi\Phi^t$$

\paragraph{} The next step is to compute the eigenvectors of $S$. However, since its dimensions could
potentially be
$\infty \times \infty$, it is desirable to avoid such a computation. We can quickly see that the centered Gramm
matrix can be expressed as $G_\lambda = \Phi^t\Phi$ and prove that $\Phi^t\Phi$ and $\Phi\Phi^t$ have the same
eigenvectors $v_i$:

\begin{align}\label{eq:grammcovar}
\Phi^t\Phi v_i &= \beta v_i \\
(\Phi\Phi^t)\Phi v_i &= \beta (\Phi v_i)
\end{align}

\paragraph{} Fortunately, $G_\lambda$ is of dimension $N \times N$, and we can compute its
eigenvectors because the dataset $X$ is finite. The expression of $G_\lambda$ is:

\begin{align*}
(G_\lambda)_{i, j} &= k\left(\varphi(x_i) - \lambda, \varphi(x_j) - \lambda\right) \\
&= k(x_i, x_j) - \underbrace{\frac{1}{N}\sum_{n = 1}^N k(x_n, x_j)}_{\text{mean of column of G}} - \underbrace{\frac{1}{N}\sum_{n = 1}^N k(x_i, x_n)}_{\text{mean of line of G}} + \underbrace{\frac{1}{N^2}\sum_{n = 1}^{N}\sum_{l = 1}^{N} k(x_n, x_l)}_{\text{mean of G}}
\end{align*}

\paragraph{} We know from equation (\ref{eq:grammcovar}) that if $(\beta, v)$ is a couple (eigenvalue, eigenvector) of $G_\lambda$, then $(\beta, \Phi v)$ is its counterpart for covariance matrix
$S$. Moreover, eigenvectors must be normalized \emph{in kernel-space} which translates in the
following:

\begin{align*}
&\| \Phi v \|^2 = 1 \\
\Rightarrow\quad &(\Phi v_i)^t(\Phi v) = 1 \\
\Rightarrow\quad &v^t\Phi^t\Phi v = 1 \\
\Rightarrow\quad &v^t G_\lambda v = 1
\end{align*}

\paragraph{} We define the scalar $\alpha$ as $\alpha = v^t G_\lambda v$. Since $v$ is an eigenvector
of $G_\lambda$ the eigenvector $v' = \frac{v}{\sqrt{\alpha}}$ is normalized in kernel-space and $\Phi
v'$ is a normalized eigenvector of $S$.


\paragraph{} The last step is to project the data onto the new space defined by the eigenvectors.
This is done by projecting $\varphi(x)$ onto $\Phi v'$ which transposes in computing the following
dot product:

\begin{align*}
\langle\varphi(x) - \lambda, \Phi v'\rangle &= \varphi(x)^t\Phi v' - \lambda^t\Phi v' \\
&= (\varphi(x) - \lambda)^t\Phi v'
\end{align*}
