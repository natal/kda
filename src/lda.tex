
\paragraph{}
The Linear Discriminant Analysis (LDA) also called Fisher's discriminant analysis (FDA) is a method that projects data living in $D$ dimensions onto a hyperplane of dimension $D'$ with $D' \leq D$ according to Fisher's criterion. 

\paragraph{}
A projection that satisfies the Fisher's criterion is a projection that maximizes the distance between the means of the two classes while minimizing the covariance within each class.

\paragraph{}
As opposed to the PCA, the LDA is a supervised method in the sense that it requires prior knowledge of the class for each vector in the dataset. LDA first appeared as a method able to work with datasets
divided in two classes, but it is easy to generalize to a multi-class configuration.

\paragraph{}
When performing a LDA on a set of data, we make the basic assumption that the data vectors follow a gaussian distribution.

\paragraph{}
We start by expressing the inter-class covariance matrix:

\begin{equation*}
S_b = \sum_{c = 1}^C N_c(\mu_c - \mu)(\mu_c - \mu)^t
\end{equation*}

Where $\mu$ is the center of the whole dataset $X$ and $\mu_c$ the center of the vectors belonging to class $c$:

\begin{align*}
\mu &= \frac{1}{N}\sum_{i = 1}^{N} x_i \\
\mu_c &= \frac{1}{N_c}\sum_{i = 1}^{N_c} x_i^c
\end{align*}

\paragraph{}
This is comparable to the classic PCA covariance computation step except we consider a dataset
exclusively composed of the center of each class.

\paragraph{}
The LDA quantifies the overall within-class scattering by summing together the covariance matrix
computed inside each class.

\begin{equation*}
S_w = \sum_{c = 1}^{C}\sum_{i = 1}^{N_c} (x^c_i - \mu_c)(x^c_i - \mu_c)^t
\end{equation*}

\paragraph{}
Fisher's criterion can be translated into an optimization problem. Since we are aiming at maximizing
between-class scattering while minimizing within-class variance, we are as a result looking for $w$ that maximizes the quotient:

\begin{equation}
\operatorname*{arg\,max}_w \frac{w^tS_bw}{w^tS_ww}
\label{eq:maximize}
\end{equation}

\paragraph{}
In order to solve the previous optimization problem, it is sufficient to
compute the $M$ biggest eigenvalues of the matrix: $ S_w^{-1}S_b $. Let
$\xoverline{w}$ be the matrix which columns are given by the eigenvectors
associated to those eigenvalues. We can finally project $x \in X$ onto $x_p$
laying on the hyperplane simply through:

\begin{equation*}
x_p = \xoverline{w}x
\end{equation*}
